\section{Support for Community Science}
\label{sec:community_science}

The Rubin Observatory has a science community that encompasses thousands of individuals worldwide, with a broad range of experience and expertise in astronomy in general, and in the analysis of optical imaging data specifically.

Rubin's model to support this diverse community to access and analyze \gls{DP1} emphasizes self-help via documentation and tutorials, and employs an open platform for asynchronous issue reporting that enables crowd-sourced solutions.
These two aspects of community support are augmented by virtual engagement activities.
In addition, Rubin supports its Users Committee to advocate on behalf of the science community, and supports the eight \gls{LSST} Science Collaborations (\secref{ssec:science_collaborations}).

All of the resources for scientists that are discussed in this section are discoverable by browsing the For Scientists pages of the Rubin Observatory website\footnote{\url{https://rubinobservatory.org/}}.

\subsection{Documentation}
\label{ssec:documentation}

The data release documentation for DP1\footnote{\url{https://dp1.lsst.io}} provides an overview of the LSSTComCam observations, detailed descriptions of the data products, and a high-level summary of the processing pipelines. 
Although much of its content overlaps significantly with this paper, the documentation is presented as a searchable, web-based resource built using Sphinx\footnote{\url{https://www.sphinx-doc.org/}}, with a focus on enabling scientific use of the data products.

\subsection{Tutorials}
\label{ssec:tutorials}

A suite of tutorials that demonstrate how to access and analyze \gls{DP1} using the RSP acompanies the DP1 release.
Jupyter Notebook tutorials are available via the ``Tutorials'' drop-down menu within the Notebook aspect of the \gls{RSP}.
Tutorials for the Portal and API aspects of the \gls{RSP} can be found in the data release documentation.

These tutorials are designed to be inclusive, accessible, clear, focused, and consistent.
Their format and contents follow a set of guidelines \citep{RTN-045} that are informed by modern standards in technical writing.


\subsection{Community Forum}
\label{ssec:forum}

The venue for all user support is the Rubin Community Forum\footnote{\url{https://community.lsst.org/}}.

Questions about any and all aspects of the Rubin data products, pipelines, and services should be posted as new topics in the Support category.
This includes beginner-level and ``naive'' questions, advanced scientific analysis questions, technical bug reports, account and data access issues, and everything in between.
The Support category of the Forum is monitored by Rubin staff, who aim to respond to all new unsolved topics within 24 hours.

The Rubin Community Forum is built on the open-source Discourse platform.
It was chosen because, for a worldwide community of ten thousand Rubin users, a traditional (i.e., closed) help desk represents a risk to Rubin science (e.g., many users with the same question having to wait for responses).
The open nature of the Forum enables self-help by letting users search for similar issues, and enables crowd-sourced problem solving (and avoids knowledge bottlenecks) by letting users help users.


\subsection{Engagement Activities}
\label{ssec:engagement}

A variety of live virtual and in-person workshops and seminars offer learning opportunities to scientists and students working with \gls{DP1}.

\begin{itemize}
\item Rubin Science Assemblies (weekly, virtual, 1 hour): alternates between hands-on tutorials based on the most recent data release and open drop-in ``office hours'' with Rubin staff.
\item Rubin Data Academy (annual, virtual, 3-4 days): an intense set of hands-on tutorials based on the most recent data release, along with co-working and networking sessions.
\item Rubin Community Workshop (annual, virtual, 5 days), a science-focused conference of contributed posters, talks, and sessions led by members of the Rubin science community and Rubin staff
\end{itemize}

For schedules and connection information, visit the For Scientists pages of the Rubin Observatory website.
Requests for custom tutorials and presentations for research groups are also accommodated.


\subsection{Users Committee}
\label{ssec:users_committee}

This committee is charged with soliciting feedback from the science community, advocating on their behalf, and recommending science-driven improvements to the \gls{LSST} data products and the Rubin Science Platform tools and services.
Community members are encouraged to attend their virtual meetings and raise issues to their attention, so they can be included in the committee's twice-yearly reports to the Rubin Observatory \gls{Director}.

The community's response to \gls{DP1} will be especially valuable input to \gls{DP2} and \gls{DR1}, and the Users Committee encourages all users to interact with them.
For a list of members and contact information, visit the For Scientists pages of the Rubin Observatory website.


\subsection{Science Collaborations}
\label{ssec:science_collaborations}

The eight \gls{LSST} Science Collaborations are independent, worldwide communities of scientists, self-organized into collaborations based on their research interests and expertise.
Members work together to apply for funding, build software infrastructure and analysis algorithms, and incorporate external data sets into their \gls{LSST}-based research.

The Science Collaborations also provide valuable advice to Rubin Observatory on the operational strategies and data products to accomplish specific science goals, and Rubin Observatory supports the collaborations via staff liaisons and regular virtual meetings with Rubin operations leadership.
