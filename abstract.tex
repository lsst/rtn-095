% Lead: Leanne
% Rewiewed by Phil 
% ApJ rules  -- one paragraph of no more  than 250 words
\begin{abstract}
We present Rubin Data Preview 1 (DP1), the first release of data from the NSF-DOE Vera C. Rubin Observatory, consisting of raw and calibrated single-epoch images, coadds, difference images, detection catalogs, and other derived data products.
DP1 is based on \nexposures science-grade optical/near-infrared exposures acquired over \nnightscomcam distinct nights by the Rubin Commissioning Camera, LSSTComCam, on the Simonyi Survey Telescope at the Summit Facility on Cerro Pach\'on, Chile during the first on-sky commissioning campaign in late 2024.
DP1 covers a total of \totalarea over \nfields roughly equally-sized non-contiguous fields, each independently observed in six broad photometric bands, $ugrizy$, spanning a range of stellar densities and latitudes and overlapping with external reference datasets.
The median image quality across all bands, measured by the FWHM of the point-spread function, is approximately 1.13 arcseconds, with the sharpest images reaching about 0.65 arcseconds.
DP1 contains approximately \nobjects distinct astrophysical objects, of which \nextendedobjects are extended in at least one band, and \nsolarsystemobjects solar system objects, of which \nnewasteroiddiscoveries are new discoveries.
DP1 is approximately \sizeinbytes in size and available to Rubin data rights holders via the Rubin Science Platform, a cloud-based environment for the analysis of petascale astronomical data.
While small compared to future LSST releases, its high quality and diversity of data support a broad range of early science investigations across all four LSST themes, providing a valuable opportunity to engage with Rubin data ahead of the start of full operations in late 2025.
\end{abstract}