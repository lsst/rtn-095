% Lead: Leanne
% Rewiewed by Phil
% ApJ rules  -- one paragraph of no more  than 250 words
% wil - it was 267 words - bought back to 250
\begin{abstract}
We present \gls{DP1}, the first data from the \gls{NSF}-\gls{DOE} Vera C. Rubin Observatory, comprising raw and calibrated single-epoch images, coadds, difference images, detection catalogs, and derived data products.
\gls{DP1} is based on \nexposures science-grade optical/near-infrared exposures acquired over \nnightscomcam distinct nights by the Rubin \gls{Commissioning} \gls{Camera}, LSSTComCam, on the Simonyi Survey Telescope at the Summit Facility on Cerro Pach\'on, Chile during the first on-sky commissioning campaign in late 2024.
\gls{DP1} covers \totalarea over \nfields roughly equally-sized non-contiguous fields, each independently observed in six broad photometric bands, $ugrizy$, spanning a range of stellar densities and latitudes and overlapping with external reference datasets.
The median image quality across all bands, measured by the \gls{FWHM} of the point-spread function, is approximately 1.13 arcseconds, with the sharpest images reaching about 0.65 arcseconds.
The 5 sigma point-source depth in each filter is  a, b, c, d, f. --- and per field?
\gls{DP1} contains approximately \nobjects distinct astrophysical objects, of which \nextendedobjects are extended in at least one band, and \nsolarsystemobjects solar system objects, of which \nnewasteroiddiscoveries are new discoveries.
\gls{DP1} is approximately \sizeinbytes in size and available to Rubin data rights holders via the Rubin \gls{Science Platform}, a \gls{cloud}-based environment for the analysis of petascale astronomical data.
While small compared to future \gls{LSST} releases, its high quality and diversity of data support a broad range of early science investigations across all four \gls{LSST} themes ahead of full operations in late 2025.
\end{abstract}
