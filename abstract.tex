% Lead: Leanne
% Rewiewed by Phil
% ApJ rules  -- one paragraph of no more  than 250 words
% wil - it was 267 words - bought back to 250
\begin{abstract}
We present Rubin Data Preview 1 (DP1), the first data from the NSF-DOE Vera C. Rubin Observatory, comprising raw and calibrated single-epoch images, coadds, difference images, detection catalogs, and ancillary data products.
DP1 is based on \nexposures science-grade optical/near-infrared exposures acquired over \nnightscomcam distinct nights by the Rubin Commissioning Camera, LSSTComCam, on the Simonyi Survey Telescope at the Summit Facility on Cerro Pach\'on, Chile in late 2024.
DP1 covers \totalarea over \nfields roughly equal-sized non-contiguous fields, each independently observed in six broad photometric bands, $ugrizy$.
The median FWHM of the point-spread function across all bands is approximately 1.13\arcsec, with the sharpest images reaching about 0.65\arcsec.
The 5$\sigma$ coadded  depths in the deepest field,  Extended Chandra Deep Field South, are u = 24.55,  g = 26.18, r = 25.96, i = 25.71, z = 25.07, y = 23.10, with the other fields no more than A mags shallower across all bands.
%The 5$\sigma$ coadded ugrizy depths in the deepest field,  Extended Chandra Deep Field South are 24.55, 26.18, 25.96, 25.71, 25.07, and 23.10, respectively,  with the other fields no more than A mags shallower across all bands
DP1 contains approximately \nobjects distinct astrophysical objects, of which \nextendedobjects are extended in at least one band, and \nsolarsystemobjects solar system objects, of which \nnewasteroiddiscoveries are new discoveries.
DP1 is approximately \sizeinbytes in size and available to Rubin data rights holders via the Rubin Science Platform, a cloud-based environment for the analysis of petascale astronomical data.
While small compared to future LSST releases, its high quality and diversity of data support a broad range of early science investigations across all four LSST  themes ahead of full operations in late 2025.
\end{abstract}

