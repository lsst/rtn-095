
\begin{deluxetable*}{lll}
\tablecaption{Descriptions of and valid values for the key data dimensions in DP1. YYYYMMDD signifies date and \# signifies a single 0--9 digit.
\label{tab:dp1_dimensions}}
\tablehead{
  \colhead{\textbf{Dimension}} &
  \colhead{\textbf{Format/Valid values}} &
  \colhead{\textbf{Description}}
}
\startdata
\texttt{day\_obs} & YYYYMMDD & \parbox[t]{11cm}{A day and night of observations that rolls over during daylight hours.} \\
\texttt{visit} & YYYYMMDD\#\#\#\#\# & \parbox[t]{11cm}{A sequence of observations processed together; synonymous with ``exposure'' in DP1.} \\
\texttt{exposure} & YYYYMMDD\#\#\#\#\# & \parbox[t]{11cm}{A single exposure of all nine ComCam detectors.} \\
\texttt{instrument} & LSSTComCam & \parbox[t]{11cm}{The instrument name.} \\
\texttt{detector} & 0--8 & \parbox[t]{11cm}{A ComCam detector.} \\
\texttt{skymap} & \texttt{lsst\_cells\_v1} & \parbox[t]{11cm}{A set of tracts and patches that subdivide the sky into rectangular regions with simple projections and intentional overlaps.} \\
\texttt{tract} & See \tabref{tab:dp1_tracts} & \parbox[t]{11cm}{A large rectangular region of the sky.} \\
\texttt{patch} & 0--99 & \parbox[t]{11cm}{A rectangular region within a tract.} \\
\texttt{physical\_filter} & \parbox[t]{4cm}{u\_02, g\_01, i\_06, r\_03, z\_03, y\_04} & \parbox[t]{11cm}{An astronomical filter.} \\
\texttt{band} & u, g, r, i, z, y & \parbox[t]{11cm}{An astronomical wave band.} \\
\enddata
\end{deluxetable*}