\section{Overview of the contents of Rubin DP1}
\label{sec:data_products}
Here we describe Rubin \gls{DP1} data products and provide summary statistics for each.
The \gls{DP1} science data products are derived from the \nvisitimages individual \gls{CCD} images taken across \nexposures exposures in the \nfields \gls{LSSTComCam} commissioning fields (\secref{ssec:pipelines_commissioning}).

The data products that comprise \gls{DP1} provide an early preview of future LSST data releases and are strongly dependent on the type and quality of the data that was collected during \gls{LSSTComCam} on-sky campaign (\secref{ssec:pipelines_commissioning}).
Consequently not all anticipated  \gls{LSST} data products, as described in the \gls{DPDD} \citep{LSE-163} were produced for the \gls{DP1} dataset.

Rubin Observatory has adopted the convention by which single-epoch detections are referred to as Sources.
By contrast, the astrophysical object associated with a given detection is referred to as an Object
\footnote{We caution that this nomenclature is not universal; for example, some surveys call ``detections'' what we call ``sources'', and use the term ``sources'' for what we call ``objects''.}.
As such, a given Object will likely have multiple associated  Sources, since it will be observed in multiple epochs.

At the highest level, the \gls{DP1} data products fall into one of five types:
\begin{itemize}
\item \textbf{Images}, including single-\gls{epoch} images, deep and template coadded images, and difference images;
\item \textbf{Catalogs} of astrophysical Sources and Objects detected and measured in the aforementioned images. We also provide the astrometric and photometric reference catalog generated from external sources that was used during processing to generate the \gls{DP1} data products;
\item \textbf{Maps}, which provide non-science-level visualizations of the data within the release. They include, for example, zoomable multi-band images and coverage maps;
\item \textbf{Ancillary data products}, including, for example, the parameters used to configure the data processing pipelines, log and processing performance files,  and \gls{calibration} data products (e.g., \gls{CTI} models, brighter-fatter kernels, etc.);
\item \textbf{Metadata} in the form of tables containing information about each visit and processed image, such as pointing, exposure time, and a range of image quality summary statistics.
\end{itemize}
While images and catalogs are expected to be the primary data products for scientific research, we also recognize the value of providing access to other data types to support investigations and ensure transparency.

To facilitate processing, Rubin \gls{DP1} uses a single skymap\footnote{A skymap is a tiling of the celestial sphere, organizing large-scale sky coverage into manageable sections for processing and analysis.} that covers the entire sky area encompassing the seven \gls{DP1} fields.
The \gls{DP1} skymap divides the entire celestial sphere into \ntotaltracts \gls{tract}s, each covering approximately \tractarea.
Each \gls{tract} is further subdivided into \npatchx$times$\npatchy equally-sized patches, with each \gls{patch} covering roughly \innerpatcharea.
Both tracts and patches overlap with their neighboring regions.
Since the \gls{LSSTComCam} only observed \totalarea of the sky during its campaign, only \ncoveredtracts out of the \ntotaltracts tracts have coverage in \gls{DP1}.
The tract identification numbers and corresponding target names for these tracts are listed in \tabref{tab:dp1_tracts}.
The size of a tract is larger than the LSSTCam field of view; however, since each observed field extends across more than one tract, each field covers multiple tracts.
%%%%% This table is auto generated from data, DO NOT EDIT
\begin{deluxetable}{lp{4.5cm}}
\caption{Tract coverage of each DP1 field. The size of a tract is larger than the LSSTCam field of view; however, since each observed field extends across more than one tract, each field covers multiple tracts.} 
\label{tab:dp1_tracts}
\tablehead{
  \colhead{\textbf{Field Code}} & \colhead{\textbf{Tract ID}} 
}
\startdata
ECDFS&5062, 5063, 5064, 4848, 4849\\
Seagull&7850, 7849, 7610, 7611\\
Rubin\_SV\_38\_7&10464, 10221, 10222, 10704, 10705, 10463\\
EDFS\_comcam&2393, 2234, 2235, 2394\\
Rubin\_SV\_095\_-25&5305, 5306, 5525, 5526\\
47\_Tuc&531, 532, 453, 454\\
Fornax\_dSph&4016, 4217, 4218, 4017\\
\enddata
\end{deluxetable}


The skymap is integral to the production of co-added images.
To create a coadded image, the processing pipeline selects all calibrated science images in a given field that meet specific quality thresholds (\secref{ssec:science_images} and \secref{ssec:coaddition}) for a given \gls{patch}, warps them onto a single consistent pixel grid for that \gls{patch}, as defined by the skymap, then coadds them.
Each individual coadd image therefore covers a single \gls{patch}.
Coadded images and the catalogs of detections from them are termed \texttt{tract}-level data products.
By contrast, \texttt{visit}-level data products are those derived from individual \gls{LSSTComCam} exposures, such as a raw image or a catalog of detections from a single calibrated image.
Most science data products (i.e., images and catalogs) in \gls{DP1} are either \texttt{tract} or \texttt{visit}--level, the main exception being the \texttt{Calibration} reference catalog.

Throughout this section, the data product names are indicated using \texttt{monospace} font.
Data products are accessed via either the \gls{IVOA} Services ( \secref{sssec:ivoa_services}) or the Data \gls{Butler} (\secref{sssec:data_butler}), or both.

\subsection{Science Images}
\label{ssec:science_images}
Science images are exposures of the night sky, as distinct from \gls{calibration} images (\secref{ssec:calibration_data}).
Although the release includes \gls{calibration} images, allowing users to reprocess the raw images if needed, this is expected to be necessary only in rare cases.
Users are strongly encouraged to start from the visit-level images provided.
The data product names shown here are those used by the Data \gls{Butler}, but the names used in the \gls{IVOA} Services differ only slightly in that they are prepended by ``\texttt{lsst.}''.

\begin{itemize}
\item \texttt{raw} images \citep{10.71929/rubin/2570310} are unprocessed data received directly from the camera.
Each \texttt{raw} corresponds to a single \gls{CCD} from a single \gls{LSSTComCam} exposure of \exposuretime duration.
Each \gls{LSSTComCam} exposure typically produces up to nine \texttt{raw}s, one per sensor in the focal plane.
However, a small number of exposures resulted in fewer than nine \texttt{raw} images due to temporary hardware issues or readout faults.

In total, \gls{DP1} includes \nraws \texttt{raw} images.
\tabref{tab:rawbreakdown} provides a summary by target and band.
A \texttt{raw} contains \nrawpixx $\times$ \nrawpixy pixels, including prescan and overscan, and occupies around \rawhdd of disk space.\footnote{Each amplifier image contains 3 and 64 columns of serial prescan and overscan pixels, respectively, and 48 rows of parallel overscan pixels, meaning a \texttt{raw} contains \nvisitimagepixx$\times$\nvisitimagepixy exposed pixels.}
The field of view of a single \texttt{raw}, excluding prescan and overscan regions, is roughly \visitimagefovx$\times$\visitimagefovy$\approx$\visitimagefov, corresponding to a plate scale of \rawplatescale.
%%%%% This table is auto generated from data, DO NOT EDIT
\setlength{\tabcolsep}{6pt}  % spacing between columns
\begin{deluxetable}{lccccccr}
\tablecaption{Number of \texttt{raw} per field and band.
\label{tab:rawbreakdown}}
\tablehead{
  \colhead{\textbf{Field Code}} & \multicolumn{6}{c}{\textbf{Band}} & \colhead{\textbf{Total}} \\
  \cline{2-7}
  & u & g & r & i & z & y & 
}
\startdata
47\_Tuc & \parbox{0.3cm}{54} & \parbox{0.3cm}{90} & \parbox{0.3cm}{288} & \parbox{0.3cm}{171} & \parbox{0.3cm}{0} & \parbox{0.3cm}{45} & 648 \\
ECDFS & \parbox{0.3cm}{387} & \parbox{0.3cm}{2070} & \parbox{0.3cm}{2133} & \parbox{0.3cm}{1455} & \parbox{0.3cm}{1377} & \parbox{0.3cm}{270} & 7692 \\
EDFS\_comcam & \parbox{0.3cm}{180} & \parbox{0.3cm}{549} & \parbox{0.3cm}{783} & \parbox{0.3cm}{378} & \parbox{0.3cm}{378} & \parbox{0.3cm}{180} & 2448 \\
Fornax\_dSph & \parbox{0.3cm}{0} & \parbox{0.3cm}{45} & \parbox{0.3cm}{225} & \parbox{0.3cm}{108} & \parbox{0.3cm}{0} & \parbox{0.3cm}{0} & 378 \\
Rubin\_SV\_095\_-25 & \parbox{0.3cm}{297} & \parbox{0.3cm}{738} & \parbox{0.3cm}{756} & \parbox{0.3cm}{207} & \parbox{0.3cm}{540} & \parbox{0.3cm}{90} & 2628 \\
Rubin\_SV\_38\_7 & \parbox{0.3cm}{0} & \parbox{0.3cm}{396} & \parbox{0.3cm}{360} & \parbox{0.3cm}{495} & \parbox{0.3cm}{180} & \parbox{0.3cm}{0} & 1431 \\
Seagull & \parbox{0.3cm}{90} & \parbox{0.3cm}{333} & \parbox{0.3cm}{387} & \parbox{0.3cm}{0} & \parbox{0.3cm}{90} & \parbox{0.3cm}{0} & 900 \\
\cline{1-8}
Total & \parbox{0.3cm}{1008} & \parbox{0.3cm}{4221} & \parbox{0.3cm}{4932} & \parbox{0.3cm}{2814} & \parbox{0.3cm}{2565} & \parbox{0.3cm}{585} & 16125 \\
\enddata
\end{deluxetable}

\item \texttt{visit\_image}s \citep{10.71929/rubin/2570311} are fully-calibrated processed images.
They have undergone instrument signature removal (\secref{ssec:isr}) and all the single frame processing steps described in \secref{ssec:single_frame_processing} which are, in summary: \gls{PSF} modeling, \gls{background} subtraction, and astrometric and photometric \gls{calibration}.
As with \texttt{raw}s, a \texttt{visit\_image} contains processed data from a single \gls{CCD} resulting from a single \exposuretime \gls{LSSTComCam} exposure.
As a consequence, a single \gls{LSSTComCam} exposure typically results in nine \texttt{visit\_image}s.
The handful of exposures with fewer than nine \texttt{raw} images also have fewer than nine \texttt{visit\_images}, but there are an additional \nsfpfails \texttt{raw} that failed processing and for which there is thus no corresponding \texttt{visit\_image}.
Almost all failures were due to challenges with astrometric fits or \gls{PSF} models in crowded fields.

In total, there are \nvisitimages \texttt{visit\_image}s in \gls{DP1}.
Each \texttt{visit\_image} comprises three images: a calibrated science image, a variance image, and a pixel-level bitmask that flags issues such as saturation, cosmic rays, or other artifacts. 
Each \texttt{visit\_image} also contains a position-dependent \gls{PSF} model, \gls{WCS} information, and various \gls{metadata} providing information about the observation and processing.
The science and variance images and the pixel mask each contain \nvisitimagepixx$\times$ \nvisitimagepixy pixels.
In total, a single \texttt{visit\_image}, including all extensions and \gls{metadata}, occupies around \visitimagehdd of disk space.

\item \texttt{deep\_coadd}s \citep{10.71929/rubin/2570313} are the product of warping and co-adding multiple \texttt{visit\_image}s covering a given \gls{patch}, as defined by the skymap.
\texttt{deep\_coadd}s are created on a per-band basis, meaning only data from exposures taken with a common filter are coadded.
As such, there are up to six \texttt{deep\_coadd}s covering each \gls{patch} -- one for each of the six \gls{LSSTComCam} bands.
The process of producing \texttt{deep\_coadd}s is described in detail in \secref{ssec:coadd_processing} but, to summarize, it involves the selection of suitable \texttt{visit\_image}s (both in terms of \gls{patch} coverage, band, and image quality), the warping of those \texttt{visit\_image}s onto a common pixel grid, and the co-adding of the warped \texttt{visit\_image}s.
To be included in a \gls{DP1} \texttt{deep\_coadd}, a \texttt{visit\_image} needed to have a \gls{PSF} \gls{FWHM} smaller than \deepcoaddmaxfwhm. Of the \nvisitimages \texttt{visit\_images}, \ndeepcoaddvisitimages satisfied this criterion and were therefore used to create \texttt{deep\_coadds}.

There are a total of \ndeepcoadds \texttt{deep\_coadd}s in \gls{DP1}.
As mentioned above, a single \texttt{deep\_coadd} covers one \gls{patch}, and includes a small amount of overlap with its neighboring \gls{patch}.
The skymap used for \gls{DP1} defines a \gls{patch} as having an on-sky area of \innerpatcharea excluding overlap, and \outerpatcharea including overlap. A single \texttt{deep\_coadd} -- including overlap -- contains \ndeepcoaddpixx $\times$ \ndeepcoaddpixy equal-sized pixels, corresponding to a platescale of \rawplatescale.
Each \texttt{deep\_coadd} contains the science image (i.e., the coadd), a variance image, and a pixel mask; all three contain the same number of pixels.
Each \texttt{deep\_coadd} also contains a position-dependent \gls{PSF} model (which is the weighted sum of the \gls{PSF} models of the input \texttt{visit\_image}s), \gls{WCS} information, plus various \gls{metadata}.

Since coadds always cover an entire \gls{patch}, it is common for a \texttt{deep\_coadd} to contain regions that were not covered by any of the selected \texttt{visit\_image}s, particularly if the \gls{patch} is on the outskirts of a field and was thus not fully observed. 
By the nature of how coadds are produced, such regions may contain seemingly valid \gls{flux} values (i.e., not necessarily zeros or \texttt{NaNs}), but will instead be flagged with the \texttt{NO\_DATA} flag in the pixel mask. 
It is therefore crucial that the pixel mask be referred to when analyzing \texttt{deep\_coadds}.

\item \texttt{template\_coadd}s \citep{10.71929/rubin/2570314} are those created to use as templates for difference imaging, i.e., the process of subtracting a template image from a \texttt{visit\_image} to identify either variable or \gls{transient} objects.\footnote{It should be noted that \texttt{template\_coadd}s are not themselves subtracted from \texttt{visit\_image}s but are, instead, warped to match the \gls{WCS} of a \texttt{visit\_image}.
It is this warped template that is subtracted from the \texttt{visit\_image} to create a difference image.
For storage space reasons, warped templates are not retained for \gls{DP1}, as they can be readily and reliably recreated from the \texttt{template\_coadd}s.}
As with \texttt{deep\_coadd}s, \texttt{template\_coadd}s are produced by warping and co-adding multiple \texttt{visit\_image}s covering a given skymap-defined \gls{patch}.
The process of building \texttt{template\_coadd}s is the same as that for \texttt{deep\_coadd}s, but the selection criteria differ between the two types of coadd.
In the case of \texttt{template\_coadd}s, one third of \texttt{visit\_image}s covering the \gls{patch} in question with the narrowest \gls{PSF} \gls{FWHM} are selected.
If one third corresponds to fewer than twelve \texttt{visit\_image}s (i.e., there are fewer than 36 \texttt{visit\_image}s covering the \gls{patch}), then the twelve \texttt{visit\_images} with the narrowest \gls{PSF} \gls{FWHM} are selected.
Finally, if there are fewer than twelve \texttt{visit\_images} covering the \gls{patch}, then all \texttt{visit\_image}s are selected. 
Of the \nvisitimages \texttt{visit\_image}s, \ntemplatecoaddvisitimages were used to create \texttt{template\_coadd}s.
This selection strategy is designed to optimize for \gls{seeing} when a \gls{patch} is well-covered by \texttt{visit\_image}s, yet still enable the production of \texttt{template\_coadd}s for poorly-covered patches.


DP1 contains a total of \ntemplatecoadds \texttt{template\_coadd}s.\footnote{The difference in the number of \texttt{deep\_coadd}s and \texttt{template\_coadd}s is due to the difference in the \texttt{visit\_image} selection criteria for each coadd.}
As with \texttt{deep\_coadd}s, a single \texttt{template\_coadd} covers a single \gls{patch}.
Since the same \texttt{skymap} is used when creating both \texttt{deep\_coadd} and \texttt{template\_coadd}s, the on-sky area and pixel count of \texttt{template\_coadd}s are the same as that of a \texttt{deep\_coadd} (see above).
Similarly, \texttt{template\_coadd}s contain the science image (i.e., the coadd), a variance image, and a pixel mask; all three contain the same number of pixels.
Also included are the \gls{PSF} model, \gls{WCS} information, and \gls{metadata}.
As is the case for \texttt{deep\_coadd}, those pixels within \texttt{template\_coadd}s that are not covered by any of the selected \texttt{visit\_image}s may still have seemingly valid values, but are indicated with the \texttt{NO\_DATA} flag within the pixel mask.

\item \texttt{difference\_image}s \citep{10.71929/rubin/2570312} are generated by the subtraction of the warped, scaled, and \gls{PSF}-matched \texttt{template\_coadd} from the \texttt{visit\_image} (see \secref{ssec:diffim_analysis}). In principle, only those sources whose \gls{flux} has changed relative to the \texttt{template\_coadd} should be apparent (at a significant level) within a \texttt{difference\_image}. In practice, however, there are numerous spurious sources present in \texttt{difference\_image}s due to unavoidably imperfect template matching.

In total, there are \ndifferenceimages \texttt{difference\_image}s in \gls{DP1}, one for each \texttt{visit\_image}.

Like \texttt{visit\_image}s, \texttt{difference\_image}s contain the science (i.e., difference) image, a variance image, and a pixel mask; all three contain the same number of pixels, which is the same as that of the input \texttt{visit\_image}. 
Also included is the \gls{PSF} model, \gls{WCS} information, and \gls{metadata}.

\item Background images contain the model \gls{background} that has been generated and removed from a science image.
% (see \secref{ssec:background_model})
\texttt{visit\_image}s, \texttt{deep\_coadd}s and \texttt{template\_coadd}s all have associated \gls{background} images.\footnote{In future data releases, \gls{background} images may be included as part of their respective science image data product.} Background images contain the same number of pixels as their respective science image, and there is one \gls{background} image for each \texttt{visit\_image}, \texttt{deep\_coadd}, and \texttt{template\_coadd}.
Difference imaging analysis also measures and subtracts a \gls{background} model, but the \texttt{difference\_background} data product is not written out by default and is not part of \gls{DP1}.

Background images are not available via the \gls{IVOA} Service; they can only be accessed via the \gls{Butler} Data Service.
\end{itemize}

\subsection{Catalogs}
\label{ssec:catalogs}
Here we describe science-ready tables produced by the science pipelines.
All but one of the catalogs described here contain data for detections in the images described in section \ref{ssec:science_images}, the exception being the \texttt{Calibration} catalog, which contains reference data obtained from previous surveys.
Observatory-produced \gls{metadata} tables are described in \secref{ssec:metadata}
Each type of catalog contains measurements for either Sources  detected in  \texttt{visit\_image}s and \texttt{difference\_image}s, or Objects detected in \texttt{deep\_coadd}s.

While the \texttt{Source}, \texttt{Object}, \texttt{ForcedSource}, \texttt{DiaSource}, \texttt{DiaObject}, and \texttt{ForcedSourceOnDiaObject} catalogs described below each differ in terms of their specific columns, in general they each contain: one or more unique identification numbers, positional information, one or more types of \gls{flux} measurements (e.g., aperture fluxes, \gls{PSF} fluxes, Gaussian fluxes, etc.), and a series of boolean flags (indicating, for example, whether the source/object is affected by saturated pixels, cosmic rays, etc.) for each source/object.
The Solar System catalogs \texttt{SSObject} and \texttt{SSSource} deviate from this general structure in that they instead contain orbital parameters for all known asteroids.
Where applicable, all measured properties are reported with their associated 1$\sigma$ uncertainties.

Since \gls{DP1} is a preview, it doesn't include all the catalogs expected in a full \gls{LSST} \gls{Data Release}. Additionally, the catalogs it does include may be missing some columns planned for future releases.
Where this is the case, we note what data are missing in the catalog descriptions that follow.

Catalog data are stored in the \gls{Qserv} database (\secref{sssec:qserv}) and are accessible via \gls{TAP}, and an online \gls{DP1} catalog \gls{schema} is available at \url{https://sdm-schemas.lsst.io/dp1.html}.
Catalog data are also accessible via the Data \gls{Butler} (\secref{sssec:data_butler}).

\begin{itemize}
\item The \texttt{Source} catalog \citep{10.71929/rubin/2570323} contains data on all sources which are, prior to deblending (\secref{sssec:coadd_processing}), detected with a greater than 5$\sigma$ significance in each individual visit.
The detections reported in the \texttt{Source} catalog have undergone deblending; in the case of blended detections, only the deblended sources are included in the \texttt{Source} catalog.
It is important to note that while the criterion for inclusion in a \texttt{Source} catalog is a $>5\sigma$ detection in a \texttt{visit\_image} prior to deblending, the positions and fluxes are reported post-deblending. 
Hence, it is possible for the \texttt{Source} catalog to contain sources whose \gls{flux}-to-error ratios -- potentially of all types (i.e., aperture \gls{flux}, \gls{PSF} \gls{flux}, etc.) -- are less than $5$.

In addition to the general information mentioned above (i.e., IDs, positions, fluxes, flags), the \texttt{Source} catalog also includes basic \gls{shape} and extendedness information.

The \texttt{Source} catalog contains data for \nsources \texttt{sources} in \gls{DP1}.

\item The \texttt{Object} catalog \citep{10.71929/rubin/2570325} contains data on all objects detected with a greater than $5\sigma$ significance in the \texttt{deep\_coadd}s.
With coadd images produced on a per-band basis, a $>5\sigma$ detection in one or more of the bands will result in an object being included in the \texttt{Object} catalog.
For cases where an object is detected at $>5\sigma$ in more than one band, a cross-matching has been performed between bands to associate an object in one band with its counterpart(s) in the other bands.
As such, unlike the \texttt{Source} catalog, the \texttt{Object} catalog contains data from multiple bands. 
The objects reported in the \texttt{Object} catalog have also undergone deblending; in the case of blended detections, only the deblended child objects are included in the catalog.
As with the \texttt{Source} catalog, the criterion for inclusion in the \texttt{Object} catalog is a $>5\sigma$ detection in one of the \texttt{deep\_coadd}s prior to deblending, yet the positions and fluxes of objects are reported post-deblending. 
Hence, it is possible for \texttt{Object} catalog to contain \texttt{objects} whose \gls{flux}-to-error ratios --- potentially of all types and in all bands --- are less than $5$.

In addition to the general information mentioned above (i.e., IDs, positions, fluxes, flags), the \texttt{Object} catalog also includes basic \gls{shape} and extendedness information.
While they may be included in future data releases, no photometric redshifts, Petrosian magnitudes \citep{1976ApJ...209L...1P}, proper motions or periodicity information are included in the \gls{DP1} object catalogs.

The \texttt{Object} catalog contains data for \nobjects objects in \gls{DP1}.

\item The \texttt{ForcedSource} catalog \citep{10.71929/rubin/2570327} contains forced \gls{PSF} photometry measurements performed on both \texttt{difference\_image}s (i.e., the \texttt{psfDiffFlux} column) and \texttt{visit\_image}s (i.e., the \texttt{psfFlux} column) at the positions of all the objects in the \texttt{Object} catalog, to allow assessment of the time variability of the fluxes.
We recommend using the \texttt{psfDiffFlux} column when generating light curves because this quantity is less sensitive to \gls{flux} from neighboring sources than \texttt{psfFlux}.
In addition to \gls{forced photometry} \gls{PSF} fluxes, a number of boolean flags are also included in the \texttt{ForcedSource} catalog.

The \texttt{ForcedSource} catalog contains a total of \nforcedsources entries across \nforcedobjects unique objects.

\item The \texttt{DiaSource} catalogs \citep{10.71929/rubin/2570317} contains data on all the sources detected at a $>5\sigma$ significance --- including those associated with known Solar System objects --- in the \texttt{difference\_image}s.
Unlike sources detected in \texttt{visit\_image}s, sources detected in difference images (hereafter, ``DiaSources'') have gone through an association step in which an attempt has been made to associate them into underlying objects called ``DiaObject''s. 
The \texttt{DiaSource} catalog consolidates all this information across multiple visits and bands. 
The detections reported in the \texttt{DiaSource} catalog have not undergone deblending.

The \texttt{DiaSource} catalog contains data for \ndiasources \texttt{DiaSources} in \gls{DP1}.

\item The \texttt{DiaObject} catalog \citep{10.71929/rubin/2570319} contains the astrophysical objects that DiaSources are associated with (i.e., the ``DiaObjects'').
The \texttt{DiaObject} catalog contains only non-Solar System Objects; Solar System Objects are, instead, recorded in the \texttt{SSObject} catalog.
When a DiaSource is identified, the \texttt{DiaObject} and \texttt{SSObject} catalogs are searched for objects to associate it with.
If no association is found, a new DiaObject is created and the DiaSource is associated to it.
Along similar lines, an attempt has been made to associate DiaObjects across multiple bands, meaning the \texttt{DiaObject} catalog, like  the \texttt{Object} catalog, contains data from multiple bands.
Since DiaObjects are typically \gls{transient} or variable (by the nature of their means of detection), the \texttt{DiaObject} catalog contains summary statistics of their fluxes, such as the mean and standard deviation over multiple epochs; users must refer to the \texttt{ForcedSourceOnDiaObject} catalog (see below) or the \texttt{DiaSource} catalog for single \gls{epoch} \gls{flux} measurements of DiaObjects.

The \texttt{DIAObject} catalog contains data for \ndiaobjects DiaObjects in \gls{DP1}.

\item The \texttt{ForcedSourceOnDiaObject} catalog \citep{10.71929/rubin/2570321} is equivalent to the \texttt{ForcedSource} catalog, but contains \gls{forced photometry} measurements obtained at the positions of all the DiaObjects in the \texttt{DiaObject} catalog.

The \texttt{ForcedSourceOnDiaObject} catalog  contains a total of \ndiaforcedsources entries across \ndiaforcedobjects unique DiaObjects.

\item The \texttt{CcdVisit} catalog \citep{10.71929/rubin/2570331} contains data for each individual processed \texttt{visit\_image}.
In addition to technical information, such as the on-sky coordinates of the central pixel and measured pixel scale, the \texttt{CcdVisit} catalog contains a range of data quality measurements, such as whole-image summary statistics for the \gls{PSF} size, zeropoint, sky \gls{background}, sky noise, and quality of astrometric solution.
It provides an efficient method to access  \texttt{visit\_image} properties without needing to access the image data.

The \texttt{CcdVisit} catalog contains entries summarizing data for all \nvisitdetectorsummaries \texttt{visit\_image}s.

\item The \texttt{SSObject} catalog \citep{10.71929/rubin/2570335}, \gls{MPCORB} and SSObject, carry information about solar system objects. 
The \gls{MPCORB} table provides the Minor Planet \gls{Center}-computed orbital elements for all known asteroids, including those that Rubin discovered. 
For DP1, the  \texttt{SSObject} catalog serves primarily to provide the mapping between the \gls{IAU} designation of an object (listed in \gls{MPCORB}), and the internal ssObjectId identifier, which is used as a key to find solar system object observations in the DiaSource and SSSource tables.
\item The \texttt{SSSource} catalog \citep{10.71929/rubin/2570333} contains data on all DiaSources that are either associated with previously-known Solar System Objects, or have been confirmed as newly-discovered Solar System Objects by confirmation of their orbital properties. 
As entries in the \texttt{SSSource} catalog stem from the \texttt{DiaSource} catalog, they have all been detected at $>5\sigma$ significance in at least one band.
The \texttt{SSSource} catalog contains data for \nsolarsystemsources Solar System Sources.

\item The \texttt{Calibration} catalog is the reference catalog that was used to perform astrometric and photometric \gls{calibration}. 
It is a whole-sky catalog built specifically for \gls{LSST}, as no single prior reference catalog had both the depth and coverage needed to calibrate \gls{LSST} data.
It combines data from multiple previous reference catalogs and contains only stellar sources.
Full details on how the \texttt{Calibration} catalog was built are provided in \cite{DMTN-277}
\footnote{In \cite{DMTN-277},  the calibration reference catalog is referred to as ``The Monster". This terminology is also carried over to the DP1 Butler.}. 
We provide a brief summary here.

For the \textit{grizy} bands, the input catalogs were (in order of decreasing priority): \gls{DES} Y6 Calibration Stars \citep{2023arXiv230501695R}; Gaia-\gls{XP} Synthetic Magnitudes \citep{2023A&A...674A..33G}; the \gls{Pan-STARRS}1 3PI Survey \citep{2016arXiv161205560C}; \gls{Data Release} 2 of the  SkyMapper survey \citep{2019PASA...36...33O}; and \gls{Data Release} 4 of the \gls{VST} \gls{ATLAS} survey \citep{2015MNRAS.451.4238S}. 
For the \textit{u}-band, the input catalogs were (in order of decreasing priority): Standard Stars from \gls{SDSS} \gls{Data Release} 16 \citep{2020ApJS..249....3A}; Gaia-\gls{XP} Synthetic Magnitudes \citep{2023A&A...674A..33G}; and synthetic magnitudes generated using \gls{SLR}, which estimates the \textit{u}-band \gls{flux} from the \textit{g}-band \gls{flux} and \textit{g-r} colors. 
This \gls{SLR} estimates were used to boost the number of \textit{u}-band reference sources, as otherwise the source density from the \textit{u}-band input catalogs is too low to be useful for the \gls{LSST}.

Only stellar sources were selected from each input catalog. 
Throughout, the \texttt{Calibration} catalog uses the \gls{DES} bandpasses for the \textit{grizy} bands and the \gls{SDSS} bandpass for the \textit{u}-band; color transformations derived from high quality sources were used to convert fluxes from the various input catalogs (some of which did not use the \gls{DES}/SDSS bandpasses) to the respective bandpasses. 
All sources from the input catalogs are matched to \textit{Gaia}-\gls{DR3} sources for robust astrometric information, selecting only isolated sources (i.e., no neighbors within 1\arcsec).

After collating the input catalogs and transforming the fluxes to the standard DES/SDSS bandpasses, the catalog was used to identify sources within a specific region of the sky. 
This process generated a set of standard columns containing positional and flux information, along with their associated uncertainties.
\end{itemize}

\subsubsection{Source Designations}\label{ssec:src_naming}
To refer to individual sources or objects from the DP1 catalogs, one should follow the LSST DP1 naming convention that has been registered with the International Astronomical Union.
Because the \texttt{Source}, \texttt{Object}, \texttt{DiaSource}, \texttt{DiaObject}, and \texttt{SSObject} tables each have their own unique IDs, their designations should differ.
In general, source designations should begin with the string ``LSST-DP1'' (denoting the Legacy Survey of Space and Time, Data Preview 1), followed by a string specifying the table that the source was obtained from.
These strings should be ``O'' (for the \texttt{Object} table), ``S'' (\texttt{Source}), ``DO'' (\texttt{DiaObject}), ``DS'' (\texttt{DiaSource}), or ``SSO'' (\texttt{SSObject}).
Following the table identifier, the designation should contain the full unique numeric identifier from the specified table (i.e., the \textit{objectId}, \textit{sourceId}, \textit{diaObjectId}, \textit{diaSourceId}, or \textit{ssObjectId}).
Each of the components of the identifier should be separated by dashes, so that the designation appears like ``LSST-DP1-TAB-123456789012345678.''
In summary, source designations should adhere to one of the following examples:

\begin{itemize}
\item Object: LSST-DP1-O-609788942606161356 (for objectId 609788942606161356)
\item Source: LSST-DP1-S-600408134082103129 (for sourceId 600408134082103129)
\item DiaObject: LSST-DP1-DO-609788942606140532 (for diaObjectId 609788942606140532)
\item DiaSource: LSST-DP1-DS-600359758253260853 (for diaSourceId 600359758253260853)
\item SSObject: LSST-DP1-SSO-21163611375481943 (for ssObjectId 21163611375481943)
\end{itemize}

Tables that were not explicitly mentioned in the description above do not have their own unique IDs, but are instead linked to one of the five tables listed above via a unique ID.
For example, the \texttt{ForcedSource} table is keyed on \textit{objectId}, \texttt{ForcedSourceOnDiaObject} uses \textit{diaObjectId}, \texttt{SSSource} is linked to \textit{diaSourceId} and \textit{ssObjectId}, and \texttt{MPCORB} uses \textit{ssObjectId}.



Maps are two-dimensional visualizations of survey data. 
In \gls{DP1}, these fall into two categories: Survey Property Maps and \gls{HiPS} Maps \citep{2015A&A...578A.114F}.

\subsection{Survey Property Maps}
Survey Property Maps \citep{10.71929/rubin/2570315} summarize how properties such as observing conditions or exposure time vary across the observed sky.
Each map provides the spatial distribution of a specific quantity at a defined sky position for each band by aggregating information from the images used to make the \texttt{deep\_coadd}.
Maps are initially created per-\gls{tract} and then combined to produce a final consolidated map.
At each sky location, represented by a spatial pixel in the \gls{HEALPix}\citep{2005ApJ...622..759G} grid, values are derived using statistical operations, such as minimum, maximum, mean, weighted mean, or sum, depending on the property.

DP1 contains \nsurveypropertymaps survey property maps.
The available maps describe total exposure times, observation epochs, \gls{PSF} size and \gls{shape}, \gls{PSF} magnitude limits, sky \gls{background} and noise levels, as well as astrometric shifts and \gls{PSF} distortions due to wavelength-dependent atmospheric \gls{DCR} effects.
They all use the dataset type  format \texttt{deep\_coadd\_<PROPERTY>\_consolidated\_map\_<STATISTIC>}.
For example, \texttt{deep\_coadd\_exposure\_time\_consolidated\_map\_sum} provides a spatial map of the total exposure time accumulated per
sky position in units of seconds.
All maps are stored in \texttt{HealSparse}\footnote{A sparse \gls{HEALPix}
representation that efficiently encodes data values on the celestial sphere. \url{https://healsparse.readthedocs.io}} format.
Survey property maps are only available via the Data \gls{Butler} (\secref{sssec:data_butler}) and  have dimensions \texttt{band} and \texttt{skymap}.

Figure \ref{fig:survey_property_maps} presents three survey property maps for exposure time, \gls{PSF} magnitude limit, and sky noise, computed for representative tracts and bands.
Because full consolidated maps cover widely separated tracts, we use clipped per-\gls{tract} views here to make the spatial patterns more discernible.
Many more survey property maps are available in the DP1 repository.
% \todo{Remove titles from plots  in figure \ref{fig:survey_property_maps}}
\begin{figure*}[hbt!]
  \centering
  \begin{subfigure}[t]{0.31\textwidth}
  \includegraphics[width=\linewidth, height=5.8cm]{deepCoadd_exposure_time_map_sum_tract10463_rband}
  \caption{Exposure time sum map for \texttt{deep\_coadd} \gls{tract} 10463, r-band in field Rubin\_SV\_38\_7}
  \end{subfigure}\hfill
  \begin{subfigure}[t]{0.31\textwidth}
  \includegraphics[width=\linewidth, height=5.8cm]{deepCoadd_psf_maglim_map_weighted_mean_tract5063_zband}
  \caption{5$\sigma$ \gls{PSF} magnitude limit weighted mean map for \texttt{deep\_coadd} \gls{tract} 5063, z-band in field ECDFS}
  \end{subfigure}\hfill
    \begin{subfigure}[t]{0.31\textwidth}
  \includegraphics[width=\linewidth, height=5.8cm]{deepCoadd_sky_noise_map_weighted_mean_tract5063_gband}
  \caption{Sky noise weighted mean map for \texttt{deep\_coadd} \gls{tract} 5063, z-band in field ECDFS}
  \end{subfigure}\hfill
\caption{Examples of survey property maps from Rubin \gls{DP1} across different bands, clipped to the boundary of a single \gls{tract} for visual clarity.}
  \label{fig:survey_property_maps}
\end{figure*}

% HiPS maps
\subsection{HiPS Maps}
\gls{HiPS} Maps \citep{2015A&A...578A.114F}, offer an interactive way to explore seamless, multi-band tiles of the sky regions covered by \gls{DP1}, allowing for smooth panning and zooming.
\gls{DP1} provides multi-band \gls{HiPS} images created by combining data from individual bands of \texttt{deep\_coadd} and \texttt{template\_coadd} images.
These images are false-color representations generated using various filter combinations for the red, green, and blue channels.
The available filter combinations include \textit{gri}, \textit{izy}, \textit{riz}, and \textit{ugr} for both \texttt{deep\_coadd} and \texttt{template\_coadd}.
Additionally, for \texttt{deep\_coadd} only, we provide color blends such as \textit{uug} and \textit{grz}.
Post-\gls{DP1}, we plan to also provide single-band HiPS images for all $ugrizy$ bands in both \gls{PNG} and \gls{FITS} formats.

\gls{HiPS} maps are only accessible through the \gls{HiPS} viewer in the \gls{RSP} Portal (\secref{ssec:rsp_portal}) and cannot be accessed via the Data \gls{Butler} (\secref{sssec:data_butler}).
All multi-band \gls{HiPS} images are provided in \gls{PNG} format.

\subsection{Metadata}
\label{ssec:metadata}
\gls{DP1} also includes \gls{metadata} about the observations, which are stored in the \texttt{Visit} table. The data it contains was produced by the observatory directly, rather than the science pipelines.
It contains technical data for each visit, such as telescope pointing, camera rotation, \gls{airmass}, exposure start and end time, and total exposure time.

\subsection{Ancillary Data Products}
\label{subsec:ancilliary}
DP1 also includes several ancillary data products. While we do not expect most users to need these, we describe them here for completeness. All the Data Products described in this section can only be accessed via the Data Butler (\secref{sssec:data_butler}).

\subsubsection{Task configuration, log, and metadata}
\gls{DP1} includes \gls{provenance}-related data products such as task logs, \gls{configuration} files, and task metadata.
Configuration files record the parameters used in each processing task, while logs and \gls{metadata} contain information output during processing. These products help users understand the processing setup and investigate potential processing failures.

\subsubsection{Calibration Data Products}
\label{ssec:calibration_data}
Calibration data products include a variety of images and models that are used to characterize and correct the performance of the camera and other system components.
These include bias, dark, and flat-field images, \gls{PTC} gains, brighter-fatter kernels \citep{2014JInst...9C3048A}, charge transfer inefficiency (\gls{CTI}) models, linearizers, and illumination corrections.
For flat-field corrections, \gls{DP1} processing used combined flats, which are averaged from multiple individual flat-field exposures to provide a stable \gls{calibration}. These \gls{calibration} products are essential inputs to \gls{ISR} (\secref{ssec:isr}). While these products are included in \gls{DP1} for transparency and completeness, users should not need to rerun ISR for their science and are advised to start with the processed \texttt{visit\_image}.
