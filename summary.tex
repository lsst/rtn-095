% Lead: Leanne, reviewed by James Mullaney
\section{Summary and Future Releases
\label{sec:summary}}

Rubin Data Preview 1 (\gls{DP1}) offers an initial look at the first on-sky data products and access services from the Vera C. Rubin Observatory. \gls{DP1} forms part of Rubin's Early Science Program, and provides the scientific community with an early opportunity to familiarize themselves with the data formats and access infrastructure for the forthcoming Legacy Survey of Space and Time (LSST).
This early release has a proprietary period of two years, during which time it is  available to Rubin data rights holders only via the cloud-based Rubin Science Platform (\gls{RSP}).

In this paper we have described the completion status of the observatory at the time of data acquisition, the commissioning campaign that forms the basis of \gls{DP1}, and the processing pipelines used to produce early versions of data products.
We provide details on the data products, their characteristics and
known issues, and describe the \gls{RSP}.

The data products described in this paper derive from observations obtained by \gls{LSSTComCam}. \gls{LSSTComCam} contains only around 5\% the number of CCDs as the full LSST Science Camera (LSSTCam), yet the DP1 dataset that it has produced will already enable a very broad range of science.
At \sizeinbytes in size, DP1 covers a total area of \totalarea and contains \nexposures single-\gls{epoch} images, \ndeepcoadds deep coadded images, \nobjects distinct astrophysical objects, including  \nnewasteroiddiscoveries  new asteroid discoveries.

While some data products anticipated from the LSST are not yet available, e.g., cell-based coadds, DP1 includes several products that will not be provided in future releases.
Notably, difference images are included in DP1 as pre-generated products; in future releases, these will instead be generated on demand via dedicated services.
The inclusion of pre-generated difference images in DP1 is feasible due to the relatively small size of the dataset, an approach that will not scale to the significantly larger data volumes expected in subsequent releases.

The \gls{RSP} is continually under development, and new functionality will continue to be deployed incrementally as it becomes available, and independent of the future data release schedule.
User query history capabilities, context-aware documentation and a bulk cutout services are just a few of the services currently under development.

Coincident with the release of DP1, Rubin Observatory begins its Science Validation Surveys with the LSST Science Camera.
This final commissioning phase will produce a dataset that will form the foundation for the second Rubin Data Preview, \gls{DP2}, expected around mid -to-late 2026.
Full operations, marking the start of the \gls{LSST}, are expected to commence by the end of 2025.